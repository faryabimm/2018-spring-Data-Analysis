%%%%%%%%%%%%%%%%%%%%%%%%%%%%%%%%%%%%%%%%%%%%%%%%%%%%%%%%%%%%%%%%%%%%%
%%%%%%%%%%%                   PROBLEM 1                   %%%%%%%%%%%
%%%%%%%%%%%%%%%%%%%%%%%%%%%%%%%%%%%%%%%%%%%%%%%%%%%%%%%%%%%%%%%%%%%%%
\subsection*{عنوان پروژه}\ \\

\begin{center}
\large
تعیین موقعیت مکانی مناسب ایجاد کسب‌و‌کار جدید با استفاده از

داده‌های مکانی شهر، توزیع جمعیت و کسب‌وکار‌های فعلی
\end{center}


\subsection*{هدف پروژه}\ \\

هدف از انجام این پروژه تولید راهکاری نرم‌افزاری داده‌محور برای ارائه‌ی مشاوره‌ی تجاری به متولیان کسب‌و‌کار‌های جدید است. محصول نهایی باید قابلیت پیشنهاد موقعیت‌های مناسب و ناب تجاری را به مشتریان داشته باشد.
در پروژه‌ی این درس  تمرکز ما بر روی تحلیل وضع مکانی فعلی کسب‌و‌کار‌ها و مراکز درمانی، تفریحی و غیره است تا بتوانیم شهود مناسبی از چیدمان فعلی  آن‌ها ارایه کنیم..
\vspace*{1cm}

\subsection*{مقدمه}\ \\

امروزه با توجه به ظهور انواع و اقسام کسب‌وکار‌های نو در کنار کسب‌وکار‌های قدیمی‌تر، ایجاد شهرک‌های مسکونی جدید و متمرکز شدن تعداد زیادی از کسب‌و کار‌های فعلی در نقاطی مشخص از شهر‌ها، میل به یافتن موقعیت مکانی مناسب برای ایجاد کسب‌وکاری جدید یا شعبه‌ی جدیدی از یک کسب‌وکار فعلی موجود به شدت احساس می‌شود. مکانی که در آن تعداد کسب‌وکار‌های مشابه و در نتیجه رقابت کمینه بوده و تقاضای مردم برای ایجاد چنین کسب‌وکاری بیشینه باشد.

از این رو نیاز به سیستمی هوشمند و داده‌محور برای پیشبینی و معرفی چنین موقعیت‌ها و ظرفیت‌هایی احساس می‌شود.

\vspace*{1cm}

\subsection*{چکیده‌ی فعالیت‌های پیش رو}\ \\

به منظور دست‌یابی به این هدف و تولید چنین سیستمی، اطلاعات کسب‌وکار‌های فعلی و همچنین توزیع مکانی‌ آنها و همینطور توزیع جمعیتی شهر استخراج خواهد شد
.
در ابتدا سعی ما بر تحلیل هرچه بهتر و دقیق‌تر و مشخص‌تر چیدمان فعلی کسب و کارها در فضای شهر تهران است تا بتوانیم اطلاعات مفید و خلاصه‌‌ای در مورد آن‌ها ارایه کنیم.
در صورتی که فاز اول با موفقیت اجرا شد سعی ما بر این است که 
و سپس 
‌های مناسب کسب‌و کار جدید با  استفاده از الگوریتم‌هایی استخراج شده و به متولیان کسب‌وکار توصیه خواهد شد. سعی می‌شود این موقعیت‌ها، با توجه به نیاز مردم با بازدید بالایی همراه باشند و اطلاعات مالی و حقوقی  تملک آن مکان نیز در نظر گرفته شود
.
در ادامه و به عنوان مسیری برای یک راهکار تجاری می‌توان سیستم را توسعه داد به گونه‌ای که داده‌هایی مانند مسیر‌های حمل و نقل و وضع ترافیک و کوتاه‌ترین مسیر‌های دسترسی به موقعیت را نیز در نظر بگیرد.

البته در این درس تمرکز اصلی ما بر روی گام اول و تحلیل هرچه دقیق‌تر داده‌های فعلی است و بخش دوم را انشالله به صورت مفصل‌تر‌ بعد‌ها در شرکتی که به کمک هم تاسیس خواهیم کرد پی می‌گیریم.

\vspace*{1cm}

\subsection*{چالش‌ها}\ \\

\begin{enumerate}
\item 
\textbf{استخراج اطلاعات مشاغل:}
\newline
برای استخراج این اطلاعات از سرویس رایگان
\lr{Google Places API}
استفاده خواهیم کرد. با استفاده از این سرویس می‌توان اطلاعات کامل ۱۵۰ هزار موقعیت را در روز
استخراج کرد.
\item 
\textbf{استخراج اطلاعات جمعیت:}
\newline
برای استخراج این اطلاعات از گزارش‌های مرکز آمار ایران و همچنین داده‌های موجود در وبگاه "\textbf{آماریستا}" استفاده خواهیم کرد.

همچنین می‌توان از کتابخانه‌ی Metadata در R برای به دست آوردن این اطلاعات استفاده کرد.
\item 
\textbf{کار با اطلاعات مکانی در :R}
\newline
برای کار با اطلاعات مکانی از package هایی مانند
maptools,
rgeos,
jsonlite,
sp
و
raster
استفاده خواهیم کرد.

\item 
\textbf{تحلیل داده‌ها:}
\newline
برای تحلیل داده‌ها و به دست آوردن نتایج مورد نظر از ابزار‌هایی همچون
spatial,
spatstat
و
spatgraphs
بهره خواهیم برد.


\item 
\textbf{نمایش اطلاعات به دست آمده:}
\newline
برای نمایش اطلاعات به دست آمده به صورت بصری از بسته‌های R ای مانند
dismo,
RgoogleMaps
و
googleVis
استفاده خواهیم کرد.


\end{enumerate}

